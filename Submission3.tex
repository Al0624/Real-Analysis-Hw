\documentclass{article}
\usepackage[utf8]{inputenc}
\usepackage{amssymb}

\title{Real Analysis Submission 3}
\author{ANDY LI}
\date{February 2023}
\begin{document}
\maketitle
\section*{Problem 2}
$s_n = \sum_{k=1}^{n} \frac{1}{k(k+1)(k+2)}$
\\Using partial fractions, we see that $$\frac{1}{k(k+2)} = \frac{2}{2(k(k+2))} = \frac{1}{2}(\frac{1}{k} - \frac{1}{k+2})$$
\\If we multiply by $\frac{1}{k+1}$ we get $$\frac{1}{2} \sum_{k=1}^{n}(\frac{1}{k(k+1)} - \frac{1}{(k+1)(k+2)}) = \frac{1}{2} \sum_{k=1}^{n}(\frac{1}{k} - \frac{1}{k+1} - \frac{1}{(k+1)(k+2)}) $$
\\The left two terms telescope to $\frac{1}{2}$.
\\Therefore we get $$\frac{1}{2}(\frac{1}{2} - \frac{1}{(k+1)(k+2)}) = \frac{1}{4}$$

\section*{Problem 3}
Prove Squeeze Theorem: Suppose that $(x_n)_{n=1}^{\infty}$, $(y_n)_{n=1}^{\infty}$, $(z_n)_{n=1}^{\infty}$ are sequences in $\mathbb{R}$ such that $x_n \leq y_n \leq z_n$ for all $n \in \mathbb{N}$ then there exists $l \in \mathbb{R}$ such that if sequence $x_n$ and $z_n$ have limit l, then $y_n$ has limit l.
\\Let $\epsilon > 0$
\\We know that for $n \geq N_x$ $$|x_n - l| < \epsilon$$
\\For $n \geq N_z$ $$|z_n - l| < \epsilon$$
\\Let $N := max\{N_x, N_z\}$
\\We also know $x_n \leq y_n \leq z_n$
\\Then by subtracting l from both sides, we get $$y_n - l \leq z_n - l < \epsilon$$
$$-\epsilon < x_n - l \leq y_n - l$$
\\Hence $$|y_n - l| < \epsilon$$ for all $n \geq N$ so $y_n$ has limit l.

\section*{Problem 4}
a) We can rewrite the Cesaro mean as $$\frac{\sum_{k=1}^{n}x_k}{n}$$ \\Suppose $(x_n)$ converges to x.
\\$$|x_n - x| < \frac{\epsilon}{2}$$ for $n \geq N_0 \in \mathbb{N}$
\\We can split the Cesaro mean into two parts: $$|\frac{\sum_{k=1}^{n} x_k}{n} - x| = \frac{\sum_{k=1}^{N_0-1} |x_k - x|}{n} + \frac{\sum_{j=N_0}^{n} |x_j - x|}{n} \leq |\frac{\sum_{k=1}^{N_0-1} |x_k - x|}{n}| + |\frac{\sum_{j=N_0}^{n} |x_j - x|}{n}|$$ 
\\We know that the $|x_j - x|$ term converges to x so we can set $$|\frac{\sum_{j=N_0}^{n} |x_j - x|}{n}| < \frac{n-N_0}{2n}\epsilon$$
\\ So $$|\frac{\sum_{k=1}^{N_0-1} |x_k - x|}{n}| + |\frac{\sum_{j=N_0}^{n} |x_j - x|}{n}| 
< |\frac{\sum_{k=1}^{N_0-1} |x_k - x|}{n}| + \frac{n-N_0}{2n}\epsilon$$
\\Since the first $N_0 - 1$ terms are bounded by some $A \in \mathbb{R}$
\\We choose $N_1 \geq N_0$ so that $\frac{A}{N_1} < \frac{\epsilon}{2}$
\\Therefore, the left term $$\frac{\sum_{k=1}^{N_0-1} |x_k - x|}{n} \leq \frac{(N_0 - 1)A}{n} < \frac{\epsilon}{2}$$
\\Hence $$|\frac{\sum_{k=1}^{n} x_k}{n} - x| \leq |\frac{\sum_{k=1}^{N_0-1} |x_k - x|}{n}| + \frac{n-N_0}{2n}\epsilon < \frac{\epsilon}{2} + \frac{\epsilon}{2} = \epsilon$$
\\Therefore, the Cesaro mean converges to the same limit x.
\\
\\b) One example is $x_n = (-1)^n$
\\$x_n$ does not converge, but $y_n = \frac{\sum_{k=1}^{n}(-1)^k}{n}$ does converge.
\end{document}