\documentclass{article}
\usepackage[utf8]{inputenc}
\usepackage{amssymb}

\title{Real Analysis Hw 2}
\author{ANDY LI}
\date{January 2023}

\begin{document}

\maketitle

\section*{Problem 1}
a) A Set B s.t $supB \leq infB$
\\By definition of sup and inf, we know that there is no upper bound of B less than supB, and there is no lower bound of B greater than infB.
\\Because supB is an upper bound, and infB is a lower bound, supB cannot be less than infB and vice versa.
\\$supB \neq infB$ because they are unique.
\\Therefore a set B such that $supB \leq infB$ does not exist.
\\
\\b) A bounded subset of $\mathbb{N}$ which contains infimum but not supremum.
\\Due to the nature of natural numbers, a bounded subset of naturals are closed. This means that their least upper and lower bound (supremum and infimum) are contained in the set.
\\
\\c) A bounded subset of $\mathbb{Q}$ which contains its infimum but not its supremum.
\\One example is $$y = -\frac{1}{x^2+1}$$

\section*{Problem 2}
Let A be a nonempty subset of $\mathbb{R}$ which is bounded below, and define -A to be the set of all -a, where $a \in A$. Prove: $infA = -sup(-A)$
\\First, we have the definition of infA $$\forall a \in A : \exists x \in \mathbb{R} : x \leq a$$
\\If we multiply -1 to both sides, we get $-x \geq -a$
\\Therefore -x is an upper bound for -A
\\Because -A is bounded above, by the AoC it has a supremum $s = sup(-A) \Rightarrow -a \leq s \leq -x$
\\By flipping signs, we also have $x \leq -s \leq a \Rightarrow -s = inf(A)$
\\Therefore $inf(A) = -s = -sup(-A)$

\section*{Problem 3}
From the density of rational numbers we have $$\forall a,b \in \mathbb{R} \exists r \in \mathbb{Q} : a < r < b$$
\\Let a = $a/\sqrt{2}$ and b = $b/\sqrt{2}$
\\We have $$a/\sqrt{2} < r < b/\sqrt{2}$$
\\Therefore $$a < r\sqrt{2} < b$$
\\We know that $r\sqrt{2}$ is a subset of irrationals
\\Therefore irrationals are dense in $\mathbb{R}$

\section*{Problem 4}
From the one direction proof, $(\alpha)^2 < 2$ is a contradiction.
\\Now we want to find a contradiction for $(\alpha)^2 > 2$
\\
\\Let us consider the set $K = \{k \in \mathbb{R} : k^2 > 2\}$
\\Let $\alpha = infK$
\\Let us assume $\alpha^2 > 2$
\\We try to find an element in K smaller than $\alpha$
\\$$(\alpha - \frac{1}{n})^2 = \alpha^2 - \frac{2\alpha}{n} + \frac{1}{n^2} > \alpha^2 - \frac{2\alpha}{n}$$ where $n \in \mathbb{N}$
\\From AP we can choose an $n_0 \in \mathbb{N}$ small enough so that $\frac{1}{n_0} > \frac{\alpha^2-2}{2\alpha}$
\\Therefore $\frac{2\alpha}{n_0} > \alpha^2-2$
\\Therefore $$(\alpha - \frac{1}{n_0})^2 > \alpha^2 - \frac{2\alpha}{n_0} > \alpha^2 - (\alpha^2 - 2) = 2$$
\\$$\Rightarrow (\alpha - \frac{1}{n_0})^2 > 2 \Rightarrow (\alpha - \frac{1}{n_0}) \in K$$
\\Which is a contradiction because $\alpha$ is a lower bound and $(\alpha - \frac{1}{n_0}) \in K < \alpha$ 
\\Hence $\alpha^2 > 2$ is a contradiction, thus proving both directions, so $$\exists \alpha \in \mathbb{R} : \alpha^2 = 2$$

\section*{Problem 5}
Let $\lim_{x \to \infty} a_n = L$ and $\lim_{x \to \infty} a_n = M$
\\$|a_n - L| < \epsilon$ for $n \geq n_1$ and $|a_n - M| < \epsilon$ for $n \geq n_2$
\\Then let $\epsilon = \frac{|L - M|}{3}$  greater than 0 because we assume $L \neq M$
\\We can take $n_0 > Max(n_1, n_2)$
\\Therefore $|L - M| = |L - a_n + a_n - M| \leq |L - a_n| + |a_n - M|$ by the triangle rule
\\We have $|L - a_n| + |M - a_n| < \epsilon + \epsilon = 2\epsilon = \frac{2\epsilon}{3} \Rightarrow |L-M| < \frac{2\epsilon}{3}$ which is a contradiction.
\\Hence, the assumption that $|L - M| \neq 0$ is false which therefore means that if a sequence converges to a limit, that limit is unique.

\end{document}