\documentclass{article}
\usepackage[utf8]{inputenc}
\usepackage{amssymb}
\usepackage{amsmath}
\title{Real Analysis Homework 8}
\author{ANDY LI}
\date{March 2023}
\begin{document}
\maketitle
\section*{Problem 1}
Given $f: A \to \mathbb{R}$ and $g: B \to \mathbb{R}$ and $f(A) \subseteq B$ s.t $(g \circ f)(x) := g(f(x))$ is defined on A.
Let $\epsilon > 0$. Since g is continuous at $f(c)$, by the epsilon delta definition of continuity, $$\forall \epsilon_1 > 0 : \exists \delta_1 > 0 : |y - f(c)| < \delta_1 \Rightarrow |g(y) - g(f(c))| < \epsilon_1$$
\\We also assume that f is continuous at c so $$\forall \epsilon_0 > 0 : \exists \delta_0 > 0: |x - c| < \delta_0 \Rightarrow |f(x) - f(c)| < \epsilon_0$$
Let $\epsilon_0 = \delta_1$.
\\By combining the definitions, we get $$|x - c| < \delta_1 \Rightarrow |g(f(x)) - g(f(c))| < \epsilon$$ for $x \in A$
\\Hence, by definition of continuity, $g \circ f$ is continuous at c.

\section*{Problem 2}
a) Prove f is continuous at 3.
\\By definition of continuity, $$\forall \epsilon > 0 : \exists \delta > 0: |x - 3| < \delta \Rightarrow |\frac{x^2 - 9}{x-3} - 6| < \epsilon$$
$$|x - 3| < \delta \Rightarrow |(x + 3) - 6| < \epsilon$$
$$|x - 3| < \delta \Rightarrow |x - 3| < \epsilon$$
\\Let $\epsilon = \delta$
\\Then $|x - 3| < \epsilon \Rightarrow |x + 3 - 6| < \epsilon$ 
\\Hence, f is continuous at 3.
\\
\\b) From part a, we prove that f is continuous at 3 by setting $\delta = \epsilon$. Hence if $\epsilon = \frac{1}{100}$, then $\delta = \frac{1}{100}$.
\\$|f(x) - 6| < 0.01 \Rightarrow |x + 3 - 6| < 0.01 \Rightarrow |x| < 3.01$ 
\\$|x - 3| < \delta \Rightarrow |3.01 - 3| < \delta \Rightarrow \delta = \frac{1}{100}$

\section*{Problem 3}
a) In order to prove continuity, we want to show that $\lim_{x \to a}f(x) = f(c)$.
\\Using the epsilon delta definition, we have $$\forall \epsilon > 0 : \exists \delta > 0 : |x - a| < \delta \Rightarrow |f(x) - f(a)| < \epsilon$$
\\Given $|f(x) - f(c)| \leq c|x - a|$, we can set $\epsilon = c\delta$
\\Therefore, we have $$|f(x) - f(a)| \leq c|x - a| < c\delta = c\frac{1}{c}\epsilon = \epsilon$$
\\Therefore by definition, f is continuous.
\\
\\b) We can write the nth recursion $f^n(y_1) = f^{n-1}(y_1) = y_n$ as $$|y_{n+1} - y_n| = |f(y_n) - f(y_{n-1}) \leq c|y_n - y_{n-1}|$$
\\We know through induction that $$|y_{n+1} - y_n| \leq c|y_n - y_{n-1}| \leq c^2|y_{n-1} - y_{n-2}| \leq ... \leq c^n|y_1 - y_0|$$
\\Hence, $|y_{n+1} - y_n| \leq c^n|y_1 - y_0|$
\\We are given that $0 < c < 1$ so $$\lim_{n \to \infty} c^n|y_1 - y_0| = 0$$
\\Hence, the sequence converges, and because the sequence converges, it is Cauchy.

\section*{Problem 4}
a) $f(x + y) = f(x) + f(y)$
\\$f(0) = f(0) + f(0) \Rightarrow f(0) = 0$
\\$0 = f(0) = f(x - x) = f(x) + f(-x) = f(x) - f(x) \Rightarrow f(x) = f(-x)$
\\
\\b) Let $k = f(1)$
\\We know that $f(1) = f(0) + f(1)$
\\For $n \in \mathbb{N}$, 
\\$f(n) = \sum_{k=0}^{n-1} (f((k+1)) - f(k)) = \sum_{k=0}^{n-1} f(1) = nf(1) = kn$
\\For $z \in \mathbb{Z}$, we need to prove the negatives because naturals show the positives. 
\\$n \in \mathbb{Z}_- \Rightarrow f(z) = -f(-z) = -(-z)f(1) = zf(1) = kz$
\\For $r \in \mathbb{Q} = \frac{p}{q}$
\\$f(r) = f(\frac{p}{q}) = pf(\frac{1}{q}) = \frac{p}{q}f(1) = rf(1) = kr$

\section*{Problem 5}
a) Let both $f(x)$ and $g(x)$ be discontinuous at 0. One example is 
\\$f(x) =$  \begin{cases} 
      1 & x \in \mathbb{Z} \\
      
      0 & Otherwise 
   \end{cases}
\\$g(x) = 1 - f(x)$
\\In this example, both $f(x)$ and $g(x)$ are discontinuous at 0, and $f + g = 1$ and $fg$ = 0 s.t $f + g$ and $fg$ are both constant and continuous.
\\
\\b) Let f be continuous at 0, and g be discontinuous at 0.
\\Now, let $h = f + g$ and assume h is continuous at 0.
\\Then we can rewrite $g = h - f$
\\We know that $h - f$ is continuous because addition or subtraction of 2 continuous functions is continuous.
\\Therefore, $g = h - f$ is continuous which is a contradiction to our assumption.
\\Hence, h is not continuous.
\end{document}