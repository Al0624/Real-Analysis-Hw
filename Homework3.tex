\documentclass{article}
\usepackage[utf8]{inputenc}
\usepackage{amssymb}

\title{Real Analysis Hw 3}
\author{ANDY LI}
\date{January 2023}

\begin{document}

\maketitle
\section*{Problem 1}
Sequences $(a_n)_{n=1}^{\infty}$ and $(b_n)_{n=1}^{\infty}$ converge to a and b in $\mathbb{R}$ respectively. For $b \neq 0$ show $(\frac{a_n}{b_n})_{n=1}^{\infty}$ converges to $\frac{a}{b}$
\\Since converging sequences are bounded, let $A > 0$ s.t $\forall n \in \mathbb{N} |a_n| < A$
\\Then let $N_a$ be large enough s.t $\forall n \geq N_a : |a_n - a| < \frac{\epsilon(|b|-1)}{2}$
\\Let $N_b$ be large enough s.t $\forall n \geq N_b : |b_n - b| < \frac{\epsilon bb_n}{2A}$
\\Then let $N := max\{N_a, N_b\}$ \\$$\forall n \geq N : |\frac{a_n}{b_n} - \frac{a}{b}| = |\frac{a_n}{b_n} - \frac{a_n}{b} + \frac{a_n}{b} - \frac{a}{b}| \leq |\frac{a_n}{b_n} - \frac{a_n}{b}| + |\frac{a_n}{b} - \frac{a}{b}|  = |\frac{a_n}{b} - \frac{a_n}{b_n}| + |\frac{a_n}{b} - \frac{a}{b}|$$
$$ = |a_n||\frac{1}{b} - \frac{1}{b_n}| + |\frac{1}{b}||a_n - a| = |a_n||\frac{b_n - b}{bb_n}|+|\frac{1}{b}||a_n-a| = |\frac{1}{b}||\frac{a_n}{b_n}||b_n-b| + |\frac{1}{b}||a_n-a|$$
$$< |\frac{A}{bb_n}|(\frac{\epsilon bb_n}{2A}) + |\frac{1}{b}|(\frac{\epsilon(|b| - 1)}{2}) < \frac{\epsilon}{2} + \frac{\epsilon}{2} = \epsilon$$
\\Hence: $|\frac{a_n}{b_n} - \frac{a}{b}| < \epsilon$ so by definition, it converges to $\frac{a}{b}$

\section*{Problem 2}
$s_n = \sum_{k=1}^{n} \frac{1}{k(k+1)(k+2)}$
\\Using partial fractions, we see that $$\frac{1}{k(k+2)} = \frac{2}{2(k(k+2))} = \frac{1}{2}(\frac{1}{k} - \frac{1}{k+2})$$
\\If we multiply by $\frac{1}{k+1}$ we get $$\frac{1}{2} \sum_{k=1}^{n}(\frac{1}{k(k+1)} - \frac{1}{(k+1)(k+2)}) = \frac{1}{2} \sum_{k=1}^{n}(\frac{1}{k} - \frac{1}{k+1} - \frac{1}{(k+1)(k+2)}) $$
\\The left two terms telescope to $\frac{1}{2}$.
\\Therefore we get $$\frac{1}{2}(\frac{1}{2} - \frac{1}{(k+1)(k+2)}) = \frac{1}{4}$$
\\Hence as n increases higher and higher, k becomes larger and larger and the right side term approaches 0. Therefore the sequence converges to $\frac{1}{4}$
\section*{Problem 3}
Prove Squeeze Theorem: Suppose that $(x_n)_{n=1}^{\infty}$, $(y_n)_{n=1}^{\infty}$, $(z_n)_{n=1}^{\infty}$ are sequences in $\mathbb{R}$ such that $x_n \leq y_n \leq z_n$ for all $n \in \mathbb{N}$ then there exists $l \in \mathbb{R}$ such that if sequence $x_n$ and $z_n$ have limit l, then $y_n$ has limit l.
\\Let $\epsilon > 0$
\\We know that for $n \geq N_x$ $$|x_n - l| < \epsilon$$
\\For $n \geq N_z$ $$|z_n - l| < \epsilon$$
\\Let $N := max\{N_x, N_z\}$
\\We also know $x_n \leq y_n \leq z_n$
\\Then by subtracting l from both sides, we get $$y_n - l \leq z_n - l < \epsilon$$
$$-\epsilon < x_n - l \leq y_n - l$$
\\Hence $$|y_n - l| < \epsilon$$ for all $n \geq N$ so $y_n$ has limit l.

\section*{Problem 4}
a) We can rewrite the Cesaro mean as $$\frac{\sum_{k=1}^{n}x_k}{n}$$ \\Suppose $(x_n)$ converges to x.
\\$$|x_n - x| < \frac{\epsilon}{2}$$ for $n \geq N_0 \in \mathbb{N}$
\\We can split the Cesaro mean into two parts: $$|\frac{\sum_{k=1}^{n} x_k}{n} - x| = \frac{\sum_{k=1}^{N_0-1} |x_k - x|}{n} + \frac{\sum_{j=N_0}^{n} |x_j - x|}{n} \leq |\frac{\sum_{k=1}^{N_0-1} |x_k - x|}{n}| + |\frac{\sum_{j=N_0}^{n} |x_j - x|}{n}|$$ 
\\We know that the $|x_j - x|$ term converges to x so we can set $$|\frac{\sum_{j=N_0}^{n} |x_j - x|}{n}| < \frac{n-N_0}{2n}\epsilon$$
\\ So $$|\frac{\sum_{k=1}^{N_0-1} |x_k - x|}{n}| + |\frac{\sum_{j=N_0}^{n} |x_j - x|}{n}| 
< |\frac{\sum_{k=1}^{N_0-1} |x_k - x|}{n}| + \frac{n-N_0}{2n}\epsilon$$
\\Since the first $N_0 - 1$ terms are bounded by some $A \in \mathbb{R}$
\\We choose $N_1 \geq N_0$ so that $\frac{A}{N_1} < \frac{\epsilon}{2}$
\\Therefore, the left term $$\frac{\sum_{k=1}^{N_0-1} |x_k - x|}{n} \leq \frac{(N_0 - 1)A}{n} < \frac{\epsilon}{2}$$
\\Hence $$|\frac{\sum_{k=1}^{n} x_k}{n} - x| \leq |\frac{\sum_{k=1}^{N_0-1} |x_k - x|}{n}| + \frac{n-N_0}{2n}\epsilon < \frac{\epsilon}{2} + \frac{\epsilon}{2} = \epsilon$$
\\Therefore, the Cesaro mean converges to the same limit x.
\\
\\b) One example is $x_n = \frac{1}{n}$

\section*{Problem 5}
Let $f_n(x) = n^2xe^{-nx}$
$$\int_{0}^{1} f_n(x)dx = n^2\int_{0}^{1}xe^{-nx}dx = n^2[-\frac{xe^{-nx}}{n} - \int_{0}^{1} -\frac{e^{-nx}}{n}dx] = n^2[-\frac{xe^{-nx}}{n} - \frac{e^{-nx}}{n^2}]_0^1$$ $$= n^2(\frac{1}{n^2}-\frac{(n+1)e^{-n}}{n^2}) = -e^{-n}-ne^{-n}+1$$
\\$$\lim_{n \to \infty} (-e^{-n} - ne^{-n} + 1) = 1$$ because both left terms converge to 0.
\\
\\Now let's consider the other side: $$lim_{n \to \infty} n^2xe^{-xn} \Rightarrow x\lim_{n \to \infty} \frac{n^2}{e^{nx}}$$
\\Using L'hopital we get: $$xlim_{n \to \infty} \frac{2}{x^2e^{nx}} = \frac{2}{x}\lim_{n \to \infty} \frac{1}{e^{nx}} = 0$$
\\Then we plug it into the integral: $$\int_0^1 (0)dx = 0$$
\\Hence $$lim_{n \to \infty} \int_0^1 f_n(x)dx = 1 \neq 0 = \int_0^1 \lim_{n \to \infty} f_n(x)dx$$
\end{document}