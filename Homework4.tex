\documentclass{article}
\usepackage[utf8]{inputenc}
\usepackage{amssymb}

\title{Real Analysis Homework 4}
\author{ANDY LI}
\date{February 2023}
\begin{document}
\maketitle

\section*{Problem 1}
a) Let $a_n$ be a bounded sequence.
\\Prove sequence $(y_n)$ defined as $y_n := sup\{a_k: k \geq n\}$ converges.
\\Since $a_n$ is bounded, by definition we have: $$|a_n| \leq M$$ for some $n \geq N \in \mathbb{N}$
\\Hence by definition of infimum and supremum, $$-M \leq inf\{a_k : k \geq n\} \leq sup\{a_k : k \geq n\} \leq M$$
\\$y_n = sup\{a_k : k \geq n\} \leq M$
\\To show that $y_n$ is decreasing, we take $$y_{n+1} = sup\{a_k : k \geq n+1\}$$
\\Hence, because $$\{a_k : k \geq n+1\} \subseteq \{a_k : k \geq n\} \Rightarrow y_{n+1} = sup\{ a_k : k \geq n+1\} \leq sup\{a_k : k \geq n\} = y_n$$
\\Therefore $y_n$ is a decreasing sequence and because it is bounded and decreasing, $y_n$ converges.
\\
\\b) Let sequence $x_n = inf\{a_k : k \geq n\}$ so that the limit of the infimum is $\lim_{n \to \infty} inf(a_n) = \lim_{n \to \infty} x_n$
\\Because $a_n$ is bounded, the infimum exists, and because of the definition of infimum, the infimum of a subset is greater than or equal to the infimum of the set. Therefore, the infimum is increasing which means that the sequence converges by MCT.
\\
\\c) By definition of infimum and supremum, $x_n \leq y_n$ for all n.
\\Let $\lim_{n \to \infty} x_n = x$ and $\lim_{n \to \infty} y_n = y$
\\Suppose $x > y$, then for $\epsilon > 0$, $x - \epsilon < x_n < x + \epsilon$ for $n \geq N$ and same for $y_n$
\\Let $\delta = x - y$ and let $\epsilon = \frac{\delta}{2}$
\\We then have $$y_n - \frac{\delta}{2} < y_n < y + \frac{\delta}{2} = x - \frac{\delta}{2} < x_n < x + \frac{\delta}{2}$$
\\From which we see that $y_n < x_n$ which contradicts the earlier statement that $x_n < y_n$
\\Hence, $\lim_{n \to \infty} x_n \leq \lim_{n \to \infty} y_n$
\\
\\d) Let $\lim_{n \to \infty} sup = \lim_{n \to \infty} inf = L$
\\By definition of infimum and supremum, $x_n \leq a_n \leq y_n$
\\Hence, by the Squeeze theorem, because $\lim_{n \to \infty} sup = \lim_{n \to \infty} inf = L$, $ \lim_{n \to \infty} a_n = L$, and therefore does exist.

\section*{Problem 2}
Let $a_n > 0$. Then both $s_n$ and $p_n$ are monotonically increasing.
\\We can rewrite $s_n = a_1 + a_2 + ... + a_n$ and $p_n = (1+a_1)(1+a_2)...(1+a_n)$
\\For all a, $1 + a \leq e^a$
\\Hence, $$p_n = (1 + a_1)(1 + a_2)...(1 + a_n) \leq e^{a_1}e^{a_2}...e^{a_n} = e^{s_n}$$
\\Hence $p_n$ is bounded above if and only if $s_n$ is bounded above.
\\Because for $a_n > 0$, both sequences are monotonically increasing, then $p_n$ converges if and only if $s_n$ converges.

\section*{Problem 3}
$\mathbb{C} := \{z = a+bi : a,b \in \mathbb{R}\}$
\\Let $\bar{z} = a - bi$ and function $|z| = z\bar{z}$
\\Assuming multiplicative rules carry over to complex numbers: $$z\bar{z} = (a+bi)(a-bi) =a^2 +abi - abi -b^2i^2 = a^2 + b^2$$
\\$|z - w| = \bar{z} - \bar{w}$

\section*{Problem 4}
$s_n = \sum_{k=1}^{n} \frac{1}{k}$
\\a) $\sum_{k=1}^{2^n} \frac{1}{k} \geq 1 + \frac{n}{2}$
\\$n = 3$ $s_n = 1 + \frac{1}{2} + \frac{1}{3} + \frac{1}{4} + \frac{1}{5} + \frac{1}{6} + \frac{1}{7} + \frac{1}{8} = \frac{761}{280} \geq \frac{5}{2}$
\\$n = 4$ $s_n = \frac{761}{280} + \frac{1}{9} + ... + \frac{1}{16} = 3.38 \geq \frac{6}{2}$
$$\sum_{k=1}^{2^{n+1}} \frac{1}{k} = \sum_{k=1}^{2^{n}} \frac{1}{k} + \sum_{2^n}^{2*2^{n}} \frac{1}{k}$$
\\We want to show that the right side term is less than $\frac{1}{2}$
\\We can see that the sum from $2^n$ to $2^n+1$ has $2^n$ number of terms. We also know that based on the function $\frac{1}{k}$ as k gets larger, the terms get smaller.
\\Hence if we take $\sum_{k=2^n}^{2*2^n} \geq \frac{2^n}{2*2^n} = \frac{1}{2}$ then we can confirm that $$\sum_{k=1}^{2^{n+1}} \frac{1}{k} \geq 1 + \frac{n}{2} + \frac{1}{2} = 1 + \frac{n+1}{2}$$
\\Hence by induction, $\sum_{k=1}^{2^n} \geq 1 + \frac{n}{2}$ for all $n \geq 1$
\\
\\b) From the induction proof above, we saw how for each increase in n, the sum increased by a number greater than $\frac{1}{2}$. As n continues growing, we can add up more and more halves and the sum never stops growing. 
\\Hence, $$lim_{n \to \infty} s_n = \infty$$

\section*{Problem 5}
Let $(a_n)_{n=1}^{\infty}$ be a monotonic decreasing sequence which converges to 0.
\\We have $\sum_{n=1}^{\infty} (-1)^{n+1}a_n$
\\Let the base case be: $$S_2 = a_1 - a_2 < S_1 = a_1$$
\\By splitting up the partial sums into odds and even parts, we have: $$S_{2m+1} = \sum_{n=1}^{2m+1} (-1)^{n+1}a_n$$ and $$S_{2m} = \sum_{n=1}^{2m} (-1)^{n+1} a_n$$
\\Let us consider $$S_{2m+1 + 1} = S_{2m+1} - a_{2m+2} + a_{2m+3}$$
\\Since $a_n$ is decreasing, $S_{2m+2} < S_{2m+1}$
\\For the odd indexes: $$S_{2m + 1} = S_{2m} - a_{2m + 1} > S_{2m}$$
\\Hence, $$S_2 < S_{2m} < S_{2m+1} < S_1$$

\end{document}