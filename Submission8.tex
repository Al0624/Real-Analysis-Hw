\documentclass{article}
\usepackage[utf8]{inputenc}
\usepackage{amssymb}
\usepackage{amsmath}
\title{Real Analysis Submission 8}
\author{ANDY LI}
\date{March 2023}
\begin{document}
\maketitle
\section*{Problem 1}
a) Given $f: A \to \mathbb{R}$ and $g: B \to \mathbb{R}$ and $f(A) \subseteq B$ s.t $(g \circ f)(x) := g(f(x))$ is defined on A.
Let $\epsilon > 0$. Since g is continuous at $f(c)$, by the epsilon delta definition of continuity, $$\forall \epsilon_1 > 0 : \exists \delta_1 > 0 : |y - f(c)| < \delta_1 \Rightarrow |g(y) - g(f(c))| < \epsilon_1$$
\\We also assume that f is continuous at c so $$\forall \epsilon_0 > 0 : \exists \delta_0 > 0: |x - c| < \delta_0 \Rightarrow |f(x) - f(c)| < \epsilon_0$$
Let $\epsilon_0 = \delta_1$.
\\By combining the definitions, we get $$|x - c| < \delta_1 \Rightarrow |g(f(x)) - g(f(c))| < \epsilon$$ for $x \in A$
\\Hence, by definition of continuity, $g \circ f$ is continuous at c.
\\
\\b) We want to show that $g(f(x_n)) \to g(f(c))$
\\First, we can assume that $(x_n) \to c$ when $c \in A$
\\From the proof above, we know that f is continuous at c. Therefore $f(x_n) \to f(c)$
\\Because g is continuous at $f(c)$, $$g(f(x_n)) \to g(f(c))$$

\section*{Problem 3}
a) In order to prove continuity, we want to show that $\lim_{x \to a}f(x) = f(c)$.
\\Using the epsilon delta definition, we have $$\forall \epsilon > 0 : \exists \delta > 0 : |x - a| < \delta \Rightarrow |f(x) - f(a)| < \epsilon$$
\\Given $|f(x) - f(c)| \leq c|x - a|$, we can set $\epsilon = c\delta$
\\Therefore, we have $$|f(x) - f(a)| \leq c|x - a| < c\delta = c\frac{1}{c}\epsilon = \epsilon$$
\\Therefore by definition, f is continuous.
\\
\\b) We can write the nth recursion $f^n(y_1) = f^{n-1}(y_1) = y_n$ as $$|y_{n+1} - y_n| = |f(y_n) - f(y_{n-1}) \leq c|y_n - y_{n-1}|$$
\\We know through induction that $$|y_{n+1} - y_n| \leq c|y_n - y_{n-1}| \leq c^2|y_{n-1} - y_{n-2}| \leq ... \leq c^n|y_1 - y_0|$$
\\Hence, $|y_{n+1} - y_n| \leq c^n|y_1 - y_0|$
\\We are given that $0 < c < 1$ so $$\lim_{n \to \infty} c^n|y_1 - y_0| = 0$$
\\Hence, the sequence converges, and because the sequence converges, it is Cauchy.
\\
\\c) Let $y = \lim_{n \to \infty} y_n$
\\We know that f is continuous, so we have that $f(y) = \lim_{n \to \infty} y_{n+1} = f(y_n)$
\\By taking n large enough, $\lim_{n \to \infty} y_{n+1} = \lim_{n \to \infty} y_n$
\\Hence, $$f(y) = \lim_{n \to \infty} y_{n+1} = \lim_{n \to \infty} y_n = y$$
\\Therefore $f(y) = y$ and by definition, y is a fixed point.
\\Additionally, we have that $f(y_n) = $
\\
\\d) From part c, we have that the sequence $(y_n)$ converges to a limit $x$ s.t y is a fixed point.
\\By definition, $|f(x) - f(y)| = |x - y|$
\\We are also given that $|f(x) - f(y)| \leq c|x - y|$
\\Because c is between 0 and 1, in order for x to satisfy both constraints, $x = y$
\\Hence, if $|f(x) - f(y)| \leq c|x - y|$ for $0 < c < 1$ then the sequence of recursions converges to a unique fixed point.


\section*{Problem 5}
a) Let both $f(x)$ and $g(x)$ be discontinuous at 0. One example is 
\\$f(x) =$  \begin{cases} 
      1 & x \in \mathbb{Z} \\
      
      0 & Otherwise 
   \end{cases}
\\$g(x) = 1 - f(x)$
\\In this example, both $f(x)$ and $g(x)$ are discontinuous at 0, and $f + g = 1$ and $fg$ = 0 s.t $f + g$ and $fg$ are both constant and continuous.
\\
\\b) Let f be continuous at 0, and g be discontinuous at 0.
\\Now, let $h = f + g$ and assume h is continuous at 0.
\\Then we can rewrite $g = h - f$
\\We know that $h - f$ is continuous because addition or subtraction of 2 continuous functions is continuous.
\\Therefore, $g = h - f$ is continuous which is a contradiction to our assumption.
\\Hence, h is not continuous. This is infeasible.
\\
\\c) One example is to let $f(x) = 0$ and $g(x)$ be any function which is discontinuous at 0.
\\In this example, $fg = 0$ and is therefore feasible.
\\
\\d) One example is $f(x) = $\begin{cases} 
      $1/2$ & x = 0 \\
      
      2 & Otherwise 
   \end{cases}
   \\In this example, $f + \frac{1}{f} =  2.5$ and is constant and continuous at 0.
\\
\\e) Let us assume that $f(x)^3$ is continuous at 0.
\\Let $f(0) \not = 0$
\\By definition of continuity, $$\forall \epsilon > 0 : \exists \delta > 0 : |x - 0| < \delta \Rightarrow |f(x)^3 - f(0)^3| < \frac{3f(0)^2\epsilon}{4}$$
\\The right side can be written as $|f(x) - f(0)| * |f(x)^2 + f(x)f(0) + f(0)^2|$
$$= (\frac{f(x)-f(0)}{2})^2 + \frac{3}{4}f(0)^2 \geq \frac{3f(0)^2}{4}$$
\\Hence, $|f(x) - f(0)| < \epsilon$ and $f(x)$ is continuous at 0.
\\Suppose $f(0) = 0$.
\\Then by definition, $$\forall \epsilon > 0 : \exists \delta > 0 : |x - 0| < \delta \Rightarrow |f(x)^3 - 0| < \epsilon^3$$
$$\Rightarrow |f(x)| < \epsilon$$ for all delta.
\\Hence $f(x)$ is continuous at 0.
\\In both $f(0) \not = 0$ and $f(0) = 0$ cases, $f(x)$ is continuous, so therefore the scenario is not feasible.
\end{document}