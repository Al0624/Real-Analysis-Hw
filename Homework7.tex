\documentclass{article}
\usepackage[utf8]{inputenc}
\usepackage{amssymb}

\title{Real Analysis Homework 7}
\author{ANDY LI}
\date{February 2023}
\begin{document}
\maketitle
\section*{Problem 1}
i) Suppose f(x) tends to L as x approaches some a. Using the epsilon delta definition: Let $\epsilon > 0$, then there exists $\delta > 0$ s.t $|x - a| < \delta \Rightarrow |f(x) - L| < \frac{\epsilon}{|k|}$.
\\We have $|kf(x) - kL| = |k||f(x)-L| < |k|\frac{\epsilon}{|k|} = \epsilon$
\\Hence i) is proven.
\\
\\ii) Suppose functions $f(x)$ and $g(x)$ tend to L and M respectively as x approaches some a, and let $(x_n)$ be a sequence that approaches some a.
\\Let $\epsilon > 0$, s.t $|f(x_n) - L| < \frac{\epsilon}{2}$ and $|g(x_n) - M| < \frac{\epsilon}{2}$ for $n > N \in \mathbb{N}$
\\Hence by triangle rule $$|f(x_n) - L + g(x_n) - M| < |f(x_n)-L| + |g(x_n) - M| < \frac{\epsilon}{2} + \frac{\epsilon}{2} = \epsilon$$
\\Hence ii) is proven.
\\
\\iii) Suppose $f(x)$ tends to L and $g(x)$ tends to M as x approaches some a.
\\Let $\epsilon > 0$, there exists $\delta > 0$ s.t $|x - a| < \delta \Rightarrow |g(x) - M| < 1 \Rightarrow |g(x)| < |M| + 1$
\\Let $$|f(x) - L| < \frac{\epsilon}{2(|M|+1)}$$ $$|g(x)-M| < \frac{\epsilon}{2(|L|+1)}$$
\\We then have: $|f(x)g(x) - LM| = |f(x)g(x) - Lg(x) + Lg(x) - LM| =$ \\$ |(f(x) - L)g(x) + L(g(x)-M)|$
$$\leq \frac{\epsilon}{2(|M|+1)}(|M|+1) + |L|\frac{\epsilon}{2(|L|+1)} < \frac{\epsilon}{2} + \frac{\epsilon}{2} = \epsilon$$
\\Hence iii) is proven.
\\

\section*{Problem 2}
a) $\lim_{x \to 2} (3x+4) = 10$
\\By definition, $$\forall \epsilon > 0 : \exists \delta > 0 : |x - 2| < \delta \Rightarrow |3x + 4 - 10| < \epsilon$$
\\We can take $|3x + 4 - 10| = |3x - 6| = 3|x-2|$
\\Let $$\epsilon = 3\delta \Rightarrow 3|x -2| < 3\delta = 3\frac{\epsilon}{3} = \epsilon$$
\\
\\b) By definition: $$\forall \epsilon > 0 : \exists \delta > 0 : |x| < \delta \Rightarrow |x^3| < \epsilon$$
\\Let $\epsilon = \delta^{\frac{1}{3}} \Rightarrow |x^3| < \delta^3 = \epsilon^{3\frac{1}{3}} = \epsilon$
\\
\\c) By definition: $$\forall \epsilon > 0 : \exists \delta > 0 : |x - 2| < \delta \Rightarrow |x^2 + x - 1 - 5| < \epsilon$$
\\Let $\epsilon = |x+3|\delta \Rightarrow |x^2 + x -6| = |(x-2)(x+3)| < |x+3|\delta = |x+3|\frac{\epsilon}{|x+3|} = \epsilon$
\\
\\d) By definition: $$\forall \epsilon > 0 : \exisits \delta > 0 : |x-3| < \delta \Rightarrow |\frac{1}{x} - \frac{1}{3}| < \epsilon$$
\\Let $\epsilon = \frac{\delta}{|3x|} \Rightarrow |\frac{1}{x} - \frac{1}{3}| = |\frac{3-x}{3x}| < \frac{\delta}{|3x|} = \frac{|3x|\epsilon}{|3x|} = \epsilon$

\section*{Problem 3}
\\Let $\epsilon > 0$. We want to find that some $\delta > 0$ exists s.t $|x - c| < \delta \Rightarrow |g(x) - L| < \epsilon$.
\\We know $\lim_{x \to c}f(x) = L \Rightarrow$ $$|x - c| < \delta_1 \Rightarrow |f(x) - L| < \epsilon$$
\\Which becomes $$L - \epsilon < f(x) < L + \epsilon$$
\\By the same process, we have: $$0< |x - c| < \delta_2 \Rightarrow L - \epsilon < h(x) < L + \epsilon$$
\\Next, we know that $f(x) \leq g(x) \leq h(x)$ for all x.
\\Hence, there exists $$0 < |x - c| < \delta_3 \Rightarrow f(x) \leq g(x) \leq h(x)$$
\\Let $\delta = \min(\delta_1,\delta_2,\delta_3)$, we then have $$L - \epsilon \leq f(x) \leq g(x) \leq h(x) < L + \epsilon$$ for $|x - c| < \delta$.
\\Hence, $|x - c| < \delta \Rightarrow |g(x) - L| < \epsilon$, so by definition of a limit: $$\lim_{x \to c} g(x) = L$$

\section*{Problem 4}
$x \in S \subseteq \mathbb{R}$ is an isolated point of S.
\\By definition: $\exists \delta > 0 : (x- \delta, x+\delta) \cap S = \{x\})$
\\For a small enough $\delta > 0$, if $x = c$, $$|x - c| < \delta \Rightarrow 0 = |f(x) - f(c)| < \epsilon$$ for all $\epsilon > 0$
\end{document}