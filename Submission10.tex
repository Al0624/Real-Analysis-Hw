\documentclass{article}
\usepackage[utf8]{inputenc}
\usepackage{amssymb}
\usepackage{amsmath}
\title{Real Analysis Submission 10}
\author{ANDY LI}
\date{April 2023}
\begin{document}
\maketitle
\section*{Problem 2}
Let $\epsilon > 0$ and $x \in [a, b]$.
\\Next, let $L_1 \in (f(a), f(x)) \CAP (f(x) - \epsilon, f(x))$
\\From the Intermediate Value Property, we can choose $c_1 \in (a, x)$ s.t $f(c_1) = L_1$
\\Hence, $|f(x) - f(c_1)| < \epsilon$
\\We have $L_2 \in (f(x), f(b)) \CAP (f(x), f(x) + \epsilon)$
\\Similarly, we can choose $c_2 \ in (x, b)$ s.t $f(c_2) = L_2$
\\Hence, $|f(c_2) - f(x)| < \epsilon$
\\We know that f is monotonically increasing, we can choose $c_1' \in (c_1, x)$ and $c_2' \in (x, c_2)$ s.t $f(x) - f(c_1') < \epsilon$ and $f(c_2') - f(x) < \epsilon$
\\We can then choose $y \in (c_1, c_2)$ s.t $|f(x) - f(y)| < \epsilon$
\\And $\delta = \min\{x - c_1, c_2 - x\}$
\\Hence, by definition, f is continuous on $[a, b]$

\section*{Problem 3}
a) for f to be continuous at 0, $\lim_{x \to 0} f_a(x) = 0$
\\Hence a can only take non-zero positive values.
\\If $x < 0$, then there is a vertical asymptote and is therefore discontinuous at 0.
\\if $x = 0$, then the limit as x approaches 0 is 1 which causes a jump discontinuity from 0 to 1.
\\
\\b) For f to be differentiable at 0, the first derivative must approach 0 as x approaches 0.
\\We can use $\lim_{x \to 0^+}\frac{f(x) - f(0)}{x - 0} = 0$
\\This simplifies into $\lim_{x \to 0^+}\frac{x^a}{x} = \lim_{x \to 0^+} x^{a - 1} = 0$
\\Hence, $a > 1$ by the same logic as part a.
\\
\\c) For f to be twice differentiable, the second derivative must approach 0 as x approaches 0.
\\We can take the derivative of the first derivative to find the second derivative. $\frac{d}{dx}ax^{a-1} = a(a-1)x^{a-2}$
\\Hence, for the second derivative to be 0 at $x = 0$, $a > 2$
\end{document}